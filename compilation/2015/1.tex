\chapter{2015.1}

\begin{enumerate}
    \item Prove that
    \begin{equation*}
        \exists x \forall y P(x, y) \leftrightarrow \forall y \exists x P(x, y)
    \end{equation*}

    (Curiously, the ``reverse" of this implication is not true. That is,

    \begin{equation*}
        \forall y \exists x P(x, y) \leftrightarrow \exists x \forall y P(x, y)
    \end{equation*}

    How do you reconcile this?)

    \item Prove the following inference rules

    \[ \frac{P \rightarrow Q, Q \rightarrow R}{P \rightarrow R} \]
    \[ \frac{P \rightarrow Q, \neg Q}{\neg P} \]
    \[ \frac{\neg P \rightarrow \neg Q}{Q \rightarrow P} \]
    
    \item Is the following also true? Prove or disprove.

    \[ \frac{\neg P \rightarrow \neg Q}{P \rightarrow Q} \]

    \item Regarding the Horses proof in section 3.2.6 \cite{mcs-lehman}, why is 
    it wrong to claim that the mistake in the proof is with $P(n)$ being false
    for $n \geq 2$ and that it assumes something false, namely, $P(n)$ in order
    to prove $P(n+1)$?

\end{enumerate}
