\problemsection{Away Goals}

\noindent In some major football championships, each match-up sees teams play in
home and in away conditions. Should there be a tie after two games, the winner is
decided by the \emph{away goals} rule: whoever scored the most goals while playing
in away territory wins that leg of the championships.

Given the results of the first leg of a match-up and the projection for the second
leg, determine the \emph{minimum} goals the \emph{first leg loser} needs in order
to win the fixture.
\\

\problemsection{Input}

\noindent Each test case is made up of three lines: the first two indicate the
score for each team in the first leg while the last one will indicate the
projected score of the first-leg winner in the second leg. The format will be

\begin{verbatim}
TN1 S1
TN2 S2
PS
\end{verbatim}

Where \verb|TN1| and \verb|TN2| are the team names and \verb|S1| and \verb|S2|
are the scores in the first leg. The team names will contain no spaces and the
scores (including the projected score) are guaranteed to be integers. The scores
will be no more than 10.
\\

\problemsection{Output}

\noindent The output for each test case is a single integer indicating the number
of goals the first-leg losing team needs to score in order to win the fixture.
\\

\problemsection{Example Input}

\begin{verbatim}
FCBarcelona 3
FCBayern 0
2
Juventus 2
RealMadrid 1
1
ParisSaintGermain 1
FCBarcelona 3
2
\end{verbatim}

\problemsection{Example Output}

\begin{verbatim}
6
3
4
\end{verbatim}
