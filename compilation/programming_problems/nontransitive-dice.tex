\problemsection{Nontransitive Dice}

\begin{definition}
A die $A$ wins against a die $B$ if, in a roll, the number shown by die $A$ is
greater than the number shown in die $B$.
\end{definition}

\begin{definition}
A die $A$ is \emph{stronger} than a die $B$ if it has greater probability of
winning against die $B$.
\end{definition}

Obviously, a conventional fair die is no stronger than any other conventional
fair die. This problem will deal with dice that are more likely to show a
certain value than others.

\textbf{Note:} Each \emph{face} of every die has an equal probability of showing
up; it is the way the values are assigned to these faces that makes them differ
from your conventional six-sided die.

Given two dice, determine if the first die ($A$) or the second die ($B$) is
stronger.
\\

\problemsection{Input}

\noindent Each test case is made up of twelve integers. The first six indicate
the values shown on the faces of die $A$ while the next six indicate the values
shown in the faces of die $B$.
\\

\problemsection{Output}

\noindent Output \verb|A| or \verb|B| if die $A$ or die $B$ is stronger. If they
are of equal strength, output \verb|tie|.
\\

\problemsection{Example Input}

\begin{verbatim}
3 3 3 3 3 3 4 4 4 4 0 0
4 4 4 4 0 0 5 5 5 1 1 1
6 6 2 2 2 2 5 5 5 1 1 1
\end{verbatim}

\problemsection{Example Output}

\begin{verbatim}
B
B
A
\end{verbatim}
